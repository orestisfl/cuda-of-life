\chapter{Αποτελέσματα}

Όλες οι μετρήσεις έγιναν στον diades για τα αρχικά πλέγματα.
Ο έλεγχος ορθότητας των υλοποιήσεων έγινε συγκρίνοντας το md5sum των αποτελεσμάτων τους.
Για κάθε αρχικό πλέγμα .bin και για όλους τους αριθμούς επαναλήψεων και οι 7 εκδόσεις παρήγαγαν ίδιο .bin αρχείο σαν αποτέλεσμα.

Από τις μετρήσεις στο \hyperref[fig:speedup]{\figurename{} \ref{fig:speedup}}
παρατηρούμε ότι οι υλοποιήσεις σε CUDA είναι μέχρι και 900 φορές γρηγορότερες σε σχέση με τον σειριακό κώδικα.
Επίσης, φαίνεται πως η bboard είναι πιο αργή για $N$ 500 (δεν διαιρείται με το 8) από την bboard1x64 αλλά στην συνέχεια είναι καλύτερη για $N$ 1000, 2000 και 4000.
Καλύτερα αποτελέσματα συνολικά έχει η bboard1x64-shared.

Στο \hyperref[fig:cuda-N]{\figurename{} \ref{fig:cuda-N}} φαίνονται τα μεγάλα οφέλη των bitfields.
Στο \hyperref[fig:cuda-iter]{\figurename{} \ref{fig:cuda-iter}} φαίνεται ότι ο χρόνος εκτέλεσης αυξάνεται γραμμικά με τον αριθμό επαναλήψεων.

\begin{figure}[h]
\centering
\subfloat[]{\label{fig:speedup:a}\includegraphics[height=0.35\textheight]{plots/speedup_iters_10.pdf}}\
\subfloat[]{\label{fig:speedup:b}\includegraphics[height=0.35\textheight]{plots/speedup_iters_100.pdf}}\
\subfloat[]{\label{fig:speedup:c}\includegraphics[height=0.35\textheight]{plots/speedup_iters_1000.pdf}}\
\caption{Επιτάχυνση σε σχέση με σειριακό κώδικα cpu.}
\label{fig:speedup}
\end{figure}

\begin{figure}[h]
\centering
\subfloat[]{\label{fig:cuda-N:a}\includegraphics[height=0.35\textheight]{plots/abs_cuda_10.pdf}}\
\subfloat[]{\label{fig:cuda-N:b}\includegraphics[height=0.35\textheight]{plots/abs_cuda_100.pdf}}\
\subfloat[]{\label{fig:cuda-N:c}\includegraphics[height=0.35\textheight]{plots/abs_cuda_1000.pdf}}\
\caption{Σύγκριση των υλοποιήσεων σε CUDA}
\label{fig:cuda-N}
\end{figure}


\begin{figure}[h]
\centering
\subfloat[]{\label{fig:cuda-iter:a}\includegraphics[width=0.5\linewidth]{plots/abs_cuda_N_500.pdf}}
\subfloat[]{\label{fig:cuda-iter:b}\includegraphics[width=0.5\linewidth]{plots/abs_cuda_N_1000.pdf}}\
\subfloat[]{\label{fig:cuda-iter:c}\includegraphics[width=0.5\linewidth]{plots/abs_cuda_N_2000.pdf}}
\subfloat[]{\label{fig:cuda-iter:d}\includegraphics[width=0.5\linewidth]{plots/abs_cuda_N_4000.pdf}}\
\caption{Σύγκριση των υλοποιήσεων σε CUDA}
\label{fig:cuda-iter}
\end{figure}